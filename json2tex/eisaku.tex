% latex uft-8
\documentclass[uplatex,dvipdfmx,a4paper,10pt,oneside,openany]{jsarticle}
\usepackage{fancyhdr}
\begin{document}
\pagestyle{fancy}
\fancyhf{準1級:完全制覇}
\fancyhead[L]{教育・育児}
\fancyhead[R]{オンライン教育の有効性}
\fancyfoot[L]{\thepage}
\fancyfoot[R]{\today}
(P.66, L.7:従来の授業にある共同体意識)従来の授業は生徒に共同体意識を与えるが、これはオンラインコースにはないものだ。ほかの生徒と顔を合わせてやりとりすることで学習の楽しさは増す。これに対して、家でオンラインコースを一人で勉強していると、生徒は孤独を感じやすい。
\vfill
(P.66, L.8:コンピュータの使用に関する問題)オンラインコースの受講には技術を使えることが必要で、このことは問題点となる可能性がある。コンピュータを使うのが得意でない人は多い。その結果、そういう人がオンラインコースを受講しても効果的に学習するのは難しい。このことは特にコースの受講を希望する高齢の受講者に当てはまる。
\vfill
(P.66, L.6:オンラインコースの多様性)オンラインコースは従来の講座よりも提供できる種類が多い。オンラインコースは世界中の人が講師になれるので、オンラインで学べる科目には独自性の高いものが多数ある。従来の学校では、限られた数の講座しか受講できない。
\vfill
(P.64, L.2:生活に合わせて受講できる授業)オンラインコースの利点は柔軟性があることだ。例えば、忙しい仕事を持つ生徒は、昼休みに、あるいは仕事を終えてから夜間にオンラインコースの講義を見ることができる。従来の授業にはこうした柔軟性はない。
\vfill
(P.64, L.1:費用の安さ)オンラインコースの利点は、従来の授業よりも通常ずっと費用が安いことだ。このことはつまり、より多くの生徒がより多様なコースを受講できるということであり、その結果、全体としてより教養を積み技能に長けた社会集団ができあがる。
\vfill
\newpage
(P.64, L.5:技術が急速に高めている有効性)技術によるオンラインコースの有効性は急速に高まっている。生徒はテレビ会議の技術を使ってグループディスカッションをすることができ、講義ビデオは何回も再生することができる。加えて、オンラインコースにはコミュニティーページがあることが多く、ここでは何百人もの生徒が互いに支援し合うことができる。
\vfill
(P.64, L.4:教師の存在が与える緊張感)生徒は教師と直接会って話さなければならないので、従来の授業で課せられる課題の方が、生徒はより真剣に取り組むだろう。オンラインコースでは、生徒はそれほど厳しく監視されていないので、コースの課題を怠りやすくなる。
\vfill
(P.64, L.3:講義以外の活動がもたらす教育効果)従来の授業の利点の一つは、興味を持つ活動に参加することで生徒が学習できることだ。例えば、生徒と教師は刺激的な議論をしたり、ともに課題に取り組んだりすることができる。こうした活動は、通常講義だけで構成されるオンラインコースよりも教育効果が高い。
\vfill
\end{document}
