% latex uft-8
\documentclass[uplatex,dvipdfmx,a4paper,10pt,oneside,openany]{jsarticle}
\usepackage{ulem}
\usepackage{fancyhdr}
\begin{document}
\pagestyle{fancy}
\fancyhf{準1級:完全制覇}
\fancyhead[L]{教育・育児}
\fancyhead[R]{オンライン教育の有効性}
\fancyfoot[L]{\thepage}
\fancyfoot[R]{\today}
(P.66, L.7:従来の授業にある共同体意識)従来の授業は生徒に共同体意識を\xout{与えるが}、これはオンラインコースにはないものだ。
\vfill
(P.66, L.8:コンピュータの使用に関する問題)オンラインコースの受講には\xout{技術を使えることが必要で、}このことは問題点となる可能性がある。
\vfill
(P.66, L.6:オンラインコースの多様性)オンラインコースは従来の\xout{講座よりも提供できる種類が多い}。
\vfill
(P.64, L.2:生活に合わせて受講できる授業)オンラインコースの利点は\xout{柔軟性があることだ。例えば}、忙しい仕事を持つ生徒は、昼休みに、
\vfill
(P.64, L.1:費用の安さ)オンラインコースの利点は、従来の授業よりも通常ずっと費用が安いことだ。このことはつまり、
\vfill
\newpage
(P.64, L.5:技術が急速に高めている有効性)技術によるオンラインコースの有効性は急速に高まっている。
\vfill
(P.64, L.4:教師の存在が与える緊張感)生徒は教師と直接会って話さなければならないので、従来の授業で課せられる課題の方が、
\vfill
(P.64, L.3:講義以外の活動がもたらす教育効果)従来の授業の利点の一つは、興味を持つ活動に参加することで生徒が学習できることだ。
\vfill
\end{document}
